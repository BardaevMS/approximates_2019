\documentclass[a4paper,12pt]{article}
\usepackage{graphicx}
\usepackage{bm,amssymb}
\usepackage{mathrsfs}
\usepackage[unicode,colorlinks=true,filecolor=blue, menucolor=black, linkcolor=black, citecolor=black,pagebackref=white]{hyperref}
\usepackage[utf8]{inputenc}
\usepackage[russian]{babel}
\usepackage{amsmath}
\usepackage{feynmp}
\usepackage{caption}
\usepackage[left=2cm,right=2cm, top=2cm,bottom=2cm,bindingoffset=0cm]{geometry}
\begin{document}
\title{Семинар по теме: <<Интегралы с параметром>>}
\maketitle

\subsection*{Ликбез}

Порой приходится иметь дело с интегралами вида:
\[
B(p,q)=\int_{0}^{1}t^{p-1}(1-t)^{q-1}dt
\]

\noindent
или интегралами, которые сводятся к интегралам такого вида подстановкой.
Это --- так называемый бета-интеграл Эйлера или просто бета-функция.
Этот интеграл удобно выражается через $\Gamma(z)$ - гамма-функцию
Эйлера, значения и свойства которой уже хорошо известны из предыдущих
семинаров; это позволяет просто получать значения этого интеграла
при различных значениях параметров:
\[
B(p,q)=\frac{\Gamma(p)\Gamma(q)}{\Gamma(p+q)}
\]

\noindent
Добавим еще одно полезное свойство гамма-функции (приводим без доказательства):
\[
\Gamma(z)\Gamma(1-z)=\frac{\pi}{\sin\pi z}
\]

\noindent
Кроме того, в задачах этого семинара будет использоваться соотношение:

$$
\frac{1}{x^{m}}	=\frac{1}{\Gamma(m)}\int_{0}^{\infty}t^{m-1}e^{-tx}dt
$$

\subsection*{Задача 1 (интегральные представления и подстановки)}

Вычислим интеграл Френеля:

\[
I=\int_{0}^{\infty}\cos x^{2}dx
\]



\subsubsection*{Решение}

Перейдем к переменной интегрирования $t=x^{2}$. Получим:

\[
I=\int_{0}^{\infty}\cos x^{2}dx=\int_{0}^{\infty}\frac{\cos t}{2\sqrt{t}}dt
\]

\noindent
Следующий шаг нетривиален. Для взятия этого интеграла удобно воспользоваться
``интегральным представлением функции $1/\sqrt{t}$''. Такие интегральные
представления - часто используемый приём, позволяющий брать определённые
интегралы; в данном случае в роли этого интегрального представления
будет интеграл Гаусса:

\[
\frac{1}{\sqrt{t}}=\frac{2}{\sqrt{\pi}}\int_{0}^{\infty}e^{-tx^{2}}dx\Rightarrow I=\frac{1}{\sqrt{\pi}}\int_{0}^{\infty}dt\int_{0}^{\infty}dxe^{-tx^{2}}\cos t
\]

\noindent
Теперь возьмем интеграл по $t$, обозначив подынтегральную функцию
как $J(x^{2})$. Тогда:

\[
J(a)=\int_{0}^{\infty}e^{-at}\cos tdt={\rm Re}\int_{0}^{\infty}e^{-at+it}dt={\rm Re}\frac{1}{a-i}=\frac{a}{a^{2}+1}
\]

\noindent
Тем самым, получаем следующий интеграл:

\begin{eqnarray*}
I & = & \frac{1}{\sqrt{\pi}}\int_{0}^{\infty}\frac{x^{2}}{1+x^{4}}dx
\end{eqnarray*}

\noindent
Приведём два способа взятия этого интеграла.


\paragraph{``Правильный'' способ}

Получившийся интеграл является интегралом от дробно-рациональной функцией
и его можно взять стандартным методом разбиения на элементарные дроби.
Он довольно громоздкий, поэтому приведём тут метод, позволяющий вычислить
интеграл проще. Сделав замену $x=\frac{1}{t}$, заметим следующее:

\[
I=\frac{1}{\sqrt{\pi}}\int_{\infty}^{0}\left(-\frac{dt}{t^{2}}\right)\frac{1/t^{2}}{1+1/t^{4}}=\frac{1}{\sqrt{\pi}}\int_{0}^{\infty}\frac{1}{1+t^{4}}dt
\]

\noindent
Беря полусумму двух представлений для интеграла $I$, получим:

\[
I=\frac{1}{2\sqrt{\pi}}\int_{0}^{\infty}\frac{1+x^{2}}{1+x^{4}}dx
\]

\noindent
Теперь можно перейти к стандарнтой переменной для интегрирования симметрических
многочленов $t=x-\frac{1}{x}$; при этом $dt=\left(1+\frac{1}{x^{2}}\right)dx$,
получим:
\[
I=\frac{1}{2\sqrt{\pi}}\int_{-\infty}^{\infty}\frac{dt}{t^{2}+2}=\frac{1}{2\sqrt{2\pi}}\left.\arctan\frac{t}{\sqrt{2}}\right|_{-\infty}^{\infty}=\frac{1}{2}\sqrt{\frac{\pi}{2}}
\]



\paragraph{``Главный'' способ}

Очень полезно научиться сводить такие интегралы к $B$-функции, о
которой было рассказано выше. Перейдем в этом интеграле к переменной
$t=\frac{1}{1+x^{4}}\Rightarrow x=\left(\frac{1-t}{t}\right)^{1/4}\Rightarrow dx=\frac{1}{4}\left(\frac{1-t}{t}\right)^{-3/4}\left(-\frac{dt}{t^{2}}\right)$.
Имеем:

\[
I=\frac{1}{\sqrt{\pi}}\int_{1}^{0}\left(\frac{1-t}{t}\right)^{1/2}\cdot t\cdot\frac{1}{4}\left(\frac{1-t}{t}\right)^{-3/4}\left(-\frac{dt}{t^{2}}\right)=\frac{1}{4\sqrt{\pi}}\int_{0}^{1}t^{-3/4}(1-t)^{-1/4}dt=\frac{1}{4\sqrt{\pi}}B\left(\frac{1}{4},\frac{3}{4}\right)
\]

\noindent
Используя приведенные выше свойства бета- и гамма-функций, получаем:

\[
B\left(\frac{1}{4},\frac{3}{4}\right)=\frac{\Gamma\left(\frac{1}{4}\right)\Gamma\left(\frac{3}{4}\right)}{\Gamma(1)}=\frac{\pi}{\sin\frac{\pi}{4}}=\pi\sqrt{2}\Rightarrow I=\frac{1}{2}\sqrt{\frac{\pi}{2}}
\]



\paragraph{Замечение }

Заметим, что этот же ответ можно было получить гораздо проще, используя
трюк с комплексными переменными и комплексным интегралом Гаусса. Нужно
лишь обратить внимание на тонкость, связанную с тем, что $\sqrt{i}=\pm e^{i\pi/4}$,
и необходимо выбрать правильный знак: 
\[
I=\int_{0}^{\infty}\cos x^{2}dx=\frac{1}{2}{\rm Re}\int_{-\infty}^{\infty}e^{ix^{2}}dx=\frac{1}{2}{\rm Re}\sqrt{\frac{\pi}{-i}}=\frac{\sqrt{\pi}}{2}{\rm Re}e^{i\pi/4}=\frac{1}{2}\sqrt{\frac{\pi}{2}}
\]



\subsection*{Задача 2 (дифференцирование и интегрирование по параметру)}

Возьмём интеграл:

\[
I(\alpha,\beta)=\int_{0}^{\infty}\frac{\ln(\alpha^{2}+x^{2})}{\beta^{2}+x^{2}}dx
\]



\subsubsection*{Решение}

Заметим, что без логарифма интеграл легко считается. Если мы продиффиренцируем
интеграл по $\alpha$, то логарифм заменится на дробь, а интегралы
с дробями считать легче:

\[
\frac{\partial I(\alpha,\beta)}{\partial\alpha}=\int_{0}^{\infty}\frac{2\alpha}{(\beta^{2}+x^{2})(\alpha^{2}+x^{2})}dx
\]

\noindent
Пусть $\alpha\neq\beta$, тогда верно разложение: 
\[
\frac{1}{(\beta^{2}+x^{2})(\alpha^{2}+x^{2})}=\frac{1}{\alpha^{2}-\beta^{2}}\left(\frac{1}{\beta^{2}+x^{2}}-\frac{1}{\alpha^{2}+x^{2}}\right)
\]

\noindent
и каждый полученый интеграл легко считается:

\[
\frac{\partial I(\alpha,\beta)}{\partial\alpha}=2\alpha\int_{0}^{\infty}\frac{1}{\alpha^{2}-\beta^{2}}\left(\frac{1}{\beta^{2}+x^{2}}-\frac{1}{\alpha^{2}+x^{2}}\right)dx=\frac{2\alpha}{\alpha^{2}-\beta^{2}}\frac{\pi}{2}\left(\frac{1}{\beta}-\frac{1}{\alpha}\right)=\frac{\pi}{(\alpha+\beta)\beta}
\]

\noindent
На самом деле, при $\alpha=\beta$, эта формула тоже верна:

\[
\int_{0}^{\infty}\frac{2\alpha}{(\alpha^{2}+x^{2})^{2}}\cdot dx=-\frac{\partial}{\partial\alpha}\int_{0}^{\infty}\frac{dx}{\alpha^{2}+x^{2}}=-\frac{\partial}{\partial\alpha}\frac{\pi}{2\alpha}=\frac{\pi}{2\alpha^{2}}
\]

\noindent
Теперь проинтегрируем полученное выражение по $\alpha$:

\[
I(\alpha,\beta)=\frac{\pi}{\beta}\ln(\alpha+\beta)+C(\beta)
\]

\noindent
Тут ``постоянная'' $C(\beta)$ - произвольная функция, которая не
зависит от $\alpha$. Чтобы найти её подставим в исходный интеграл
$\alpha=0$:

\[
I\left(0,\beta\right)=2\int_{0}^{\infty}\frac{\ln x}{\beta^{2}+x^{2}}dx=\left|\begin{array}{c}
x=\beta t\\
dx=\beta dt
\end{array}\right|=2\int_{0}^{\infty}\frac{\ln t+\ln\beta}{\beta(1+t^{2})}dt=\frac{\pi\ln\beta}{\beta}+2\int_{0}^{\infty}\frac{\ln t}{1+t^{2}}dt
\]

\noindent
Последний интеграл равен нулю. Это просто показать, сделав замену $t=e^{z}$:

\[
\int_{0}^{\infty}\frac{\ln t}{1+t^{2}}dt=\int_{-\infty}^{\infty}\frac{z}{1+e^{2z}}e^{z}dz=\int_{-\infty}^{\infty}\frac{z}{e^{-z}+e^{z}}dz=0
\]

\noindent
Итого $I(0,\beta)=\frac{\pi\ln\beta}{\beta}$ $\Rightarrow C(\beta)\equiv0$.
Стоит заметить, что в решении мы неявно пользовались тем, что $\alpha,\beta>0$;
но поскольку исходный интеграл очевидным образом не чувствителен к
изменению их знака, то ответ в общем виде записывается как:

\[
I\left(\alpha,\beta\right)=\frac{\pi}{|\beta|}\ln(|\alpha|+|\beta|)
\]



\subsection*{Задача 3 (составление дифференциальных уравнений)}

Вычислим интегралы Лапласа:
\[
I_{1}(a,\omega)=\int_{-\infty}^{\infty}\frac{\cos(\omega x)}{x^{2}+a^{2}}dx
\]


\[
I_{2}(a,\omega)=\int_{-\infty}^{\infty}\frac{x\sin(\omega x)}{x^{2}+a^{2}}dx
\]



\subsubsection*{Решение}

Обезразмерим интеграл, перейдя к переменной интегрирования $x=at$
и введя ``безразмерный'' параметр $\alpha=\omega a$; для определенности
далее будем считать, что $\alpha>0$. Получим:

\[
I_{1}\left(a,\omega\right)=\int_{-\infty}^{\infty}\frac{\cos\alpha x}{x^{2}+1}\cdot\frac{dx}{a}=\frac{1}{a}J(\alpha)
\]

\noindent
Возьмем производную $J(\alpha)$ по $\alpha$:

\[
\frac{\partial}{\partial\alpha}J(\alpha)=\frac{\partial}{\partial\alpha}\int_{-\infty}^{\infty}\frac{\cos\alpha x}{x^{2}+1}dx=-\int_{-\infty}^{\infty}\frac{x\sin\alpha x}{x^{2}+1}dx
\]

\noindent
Получившийся интеграл сходится, однако производная от него уже расходится.
В таком случае используем такой трюк:

\[
\frac{\partial}{\partial\alpha}J(\alpha)=-\int_{-\infty}^{\infty}\frac{x^{2}\sin\alpha x}{x^{2}+1}\cdot\frac{dx}{x}=-\int_{-\infty}^{\infty}\left[1-\frac{1}{x^{2}+1}\right]\frac{\sin\alpha x}{x}dx=-\pi+\int_{-\infty}^{\infty}\frac{\sin(\alpha x)}{x^{2}+1}\cdot\frac{dx}{x}
\]

\noindent
Тут мы воспользовались табличным значением интеграла $\int_{-\infty}^{\infty}\frac{\sin x}{x}dx=\pi$.
Теперь можно вычислять вторую производную, так как получающийся интеграл
сходится:

\[
\frac{\partial^{2}}{\partial\alpha^{2}}J(\alpha)=\frac{\partial}{\partial\alpha}\int_{-\infty}^{\infty}\frac{\sin(\alpha x)}{x^{2}+1}\cdot\frac{dx}{x}=\int_{-\infty}^{\infty}\frac{\cos(\omega x)}{x^{2}+a^{2}}\cdot dx=J(\alpha)
\]

\noindent
Мы получили замкнутое дифференциальное уравнение на функцию $J(\alpha)$.
Это уравнение линейно и его коэффициенты постоянны, поэтому оно
решается с помощью подстановки $J(\alpha)=e^{\lambda\alpha}$. Такая
подстановка приводит к алгебраическому уравнению на $\lambda$: $\lambda^{2}=1\Rightarrow\lambda=\pm1$
и, следовательно, общее решение уравнения записывается как:

\[
J(\alpha)=C_{1}e^{\alpha}+C_{2}e^{-\alpha}
\]

\noindent
Константа $C_{1}$ должна быть положена равной нулю. Это связано с
тем, что исходный интеграл, очевидно, ограничен: $\left|J\left(\alpha\right)\right|\leq\left|\int_{-\infty}^{\infty}\frac{1}{x^{2}+1}dx\right|=\pi$.
Константу $C_{2}$ можно найти из значения интеграла при $\alpha=0\Rightarrow J(0)=\pi$.
Значит, наш интеграл записывается как:

\[
J(\alpha)=\pi e^{-\alpha}
\]

\noindent
Ответ был получен в предположении $\alpha>0$. Поскольку исходный
интеграл зависит лишь от модулей параметров $a$ и $\omega$, то ответ
в общем виде записывается как:

\[
I_{1}(a,\omega)=\frac{\pi}{\left|a\right|}e^{-\left|a\omega\right|}
\]

\noindent
Кроме того, заметим, что:
\[
I_{2}(a,\omega)=-\frac{\partial}{\partial\omega}I_{1}(a,\omega)=\pi e^{-\left|\alpha\omega\right|}{\rm sign}\omega
\]



\subsection*{Задача 4 (экспоненциальная регуляризация)}

Используя экспоненциальную регуляризацию, найти ``сумму'' ряда из
натуральных чисел 
\[
\sum_{n=1}^{\infty}n
\]



\paragraph{Указание}

Суть экспоненциальной регуляризации сводится к домножению на $e^{-\varepsilon n}$
и рассмотрению поведения функции при $\varepsilon\to0$.


\subsubsection*{Решение}

Нам необходимо рассмотреть следующий ряд:
\[
S(\varepsilon)=\sum_{n=1}^{\infty}ne^{-\varepsilon n}
\]

\noindent
Этот ряд можно представить как производную по $\varepsilon$ от геометрической
прогрессии:
\[
S(\varepsilon)=-\frac{\partial}{\partial\varepsilon}\sum_{n=1}^{\infty}e^{-\varepsilon n}=-\frac{\partial}{\partial\varepsilon}\frac{e^{-\varepsilon}}{1-e^{-\varepsilon}}=\frac{e^{\varepsilon}}{(e^{\varepsilon}-1)^{2}}\cdot
\]

\noindent
А теперь разложимся по $\varepsilon$, чтобы взять предел $\varepsilon\to0$:

\begin{multline*}
S\left(\varepsilon\right)=\left(\frac{1}{1+\varepsilon+\frac{1}{2}\varepsilon^{2}+\frac{1}{6}\varepsilon^{3}-1+o(\varepsilon^{3})}\right)^{2}\left(1+\varepsilon+\frac{\varepsilon^{2}}{2}+o(\varepsilon^{2})\right)=\\
=\frac{1}{\varepsilon^{2}}\left(1+\frac{\varepsilon}{2}+\frac{\varepsilon^{2}}{6}+o(\varepsilon^{2})\right)^{-2}\left(1+\varepsilon+\frac{\varepsilon^{2}}{2}+o(\varepsilon^{2})\right)=\\
=\frac{1}{\varepsilon^{2}}\left(1-\varepsilon-\frac{\varepsilon^{2}}{3}+\frac{3\varepsilon^{2}}{4}+o(\varepsilon^{2})\right)\left(1+\varepsilon+\frac{\varepsilon^{2}}{2}+o(\varepsilon^{2})\right)=\frac{1}{\varepsilon^{2}}-\frac{1}{12}+o(1)
\end{multline*}



\paragraph{Замечение}

Дзета-функция Римана при ${\rm Re}\:z>1$ определяется как сумма ряда
$\zeta(z)=\sum_{n=1}^{\infty}n^{-z}$. Используя экспоненциальную
регуляризацию, мы, на самом деле, получили значение $\zeta(-1)=-\frac{1}{12}$.
Эту функцию можно доопределить (единственным образом) и для ${\rm Re}\:z<1$
так, чтобы полученная функция получилась аналитической (то есть имела
все производные вплоть до бесконечного порядка везде, кроме некоторого
набора особых точек); и значение в $z=-1$ получено именно в смысле
аналитического продолжения. Этот результат означает, что (в определенном
смысле) сумма всех натуральных чисел равна $-\frac{1}{12}$.

\subsection*{*Задача 5 (1D эффект Казимира)}

Эта задача призвана прояснить физический смысл процедуры регуляризации.

\noindent
Рассмотрим металлические пластины, находящиеся на расстоянии $a\gg r$ друг от друга в вакууме (здесь $r$ - радиус пластин). Согласно представлениям квантовой электродинамики, вакуум - набор квантов электромагнитного поля (фотонов), находящихся в состоянии наименьшей энергии. Однако, из-за <<квантовости>> фотонов их минимальная энергия не $0$, а определяется формулой $E_{k}=\frac{1}{2}\hslash\omega_k=\frac{1}{2}\hslash ck$, $k$-волновое число. К сожалению, в полной мере обсудить её происхождение на уровне 1 курса невозможно (её можно понимать как следствие соотношения неопределённости Гейзенберга на координату и импульс). Из-за наличия такого <<моря>> фотонов по разные стороны от пластин возникает сила взаимодействия между пластинами (как разность давлений справа и слева). Необходимо найти её.

\subsubsection*{Решение}
Вычислим энергию данной системы, а затем и силу взаимодействия по формуле: $F=-\partial E/\partial a$. Как было отмечено выше, энергия фотона есть $E_{k}=\frac{1}{2}\hslash\omega_k=\frac{1}{2}\hslash ck$.

\noindent
Полная энергия (в пространстве между двумя пластинами, т.е. в цилиндре радиуса $r$ и длины $a$) равна:
$$
E=\sum_{k,\alpha}\frac{1}{2}\hslash\omega_{k,\alpha}=\frac{\hslash c}{2}2\sum_{n=0}^{\infty}\frac{n\pi}{a}
$$
Здесь $\alpha$ - индекс, нумерующий поляризации фотонов (их всего 2, отсюда и дополнительная двойка). Волновое число $k$ стало дискретным из-за того, что в металле электрическое поле равно нулю - на самих пластинах поле зануляется.

\noindent
Полученная сумма расходится. Что это может означать? Определённо, что-то пошло не так. Стандартная интерпретация этого явления такова: имеющаяся у нас теория применима только для малых энергий, а что происходит на больших, мы не понимаем. На самом деле физическая величина должна быть конечной. Это значит, что <<правильная>> энергия должна вычисляться по формуле:
$$
E=\frac{\hslash c}{2}2\sum_{n=0}^{\infty}\frac{n\pi}{a}f(n)
$$
Здесь $f(n)\to0$ при $n\to\infty$, а на не очень больших $n$ $f(n)\approx1$. Казалось бы, в таком виде ничего посчитать нельзя - нам важно знать явный вид $f(n)$. Однако, можно рассмотреть разницу энергий в данном объёме чистого вакуума и вакуума с пластинами. Найдём энергию вакуума в данном объёме. Чистый вакуум эквивалентен бесконечно удалённым пластинам, и в выражении для энергии сумма заменится на интеграл:
$$
\sum_{k}f(k)=\sum_{n=0}^{\infty}f(n)\frac{\pi}{a}\frac{a}{\pi}=\frac{a}{\pi}\sum_{k}f(k)\Delta k=\frac{a}{\pi}\int_0^{\infty}dkf(k)=a\int\frac{dk}{2\pi}f(k)
$$
В последнем переходе использовано свойство изотропии $f$ (зависимость только от энергии, но не от направления импульса). Для энергии вакуума получаем:
$$
E_{vac}=\frac{\hslash c}{2}2 a\int_{-\infty}^{+\infty}\frac{dk}{2\pi}f(k)k= \frac{\hslash c}{2}2\int_0^{\infty}\frac{n\pi}{a}f(n)dn
$$
Здесь $k=\frac{\pi n}{a}$. 
Теперь вычислим разность энергий:
$$
\Delta E=\frac{\hslash c\pi}{a}\left(\sum_{n=0}^{\infty}nf(n)-\int_0^{\infty}dn nf(n)\right)
$$
Вспомним формулу Эйлера-Пуассона (была на лекции Я.В.Фоминова про метод перевала):
$$
\sum_{n=0}^{\infty}nf(n)=\int_{0}^{\infty}nf(n)dn-\sum_{k=1}^{\infty}\frac{B_{2k}}{(2k)!}\left(nf(n)\right)^{(2k-1)}\vert_{n=0}
$$
Здесь $B_{2k}$ - числа Бернулли. В частности, $B_2=\frac{1}{6}$, $B_4=-\frac{1}{30}$. Из физических предположений разумно потребовать, чтобы $f(0)=1$, $f^{'}(0)\approx0$, и так же для остальных производных (т.е. для малых энергий наша теория работает хорошо). Тогда получаем:
$$
\sum_{n=0}^{\infty}nf(n)-\int_0^{\infty}dn nf(n)=-\frac{1}{12}
$$
Таким образом, от функции $f$ ответ не зависит. За это и ведётся борьба при регуляризации - за <<обрезание незнания>> при больших импульсах. Другими словами, мы хотим выразить ответ только через характеристики системы при малых энергиях. Заметим, что та же $-1/12$ была получена в задаче 4 из экспоненциальной регуляризации (там была явно задана $f(n)=e^{-\varepsilon n}$).

\noindent
Итак, была получена формула для $\Delta E$:
$$
\Delta E=-\frac{1}{12}\frac{\pi\hslash c}{a}
$$
Дифференцируя, находим ответ:
$$
F=-\frac{\hslash\pi c}{12 a^2}
$$
Знак минус соответствует притяжению пластин. Как бы неожиданно это ни звучало, эффект Казимира необходимо учитывать в современной технике.\footnote{Astrid Lambrecht 2002 Phys. World \textbf{15} (9) 29}
\subsection*{Задачи для домашнего решения:}

\noindent \textbf{Упражнение 1}

\noindent Вычислите интеграл 

\begin{equation}
J(y)=\int_{0}^{\pi/2}\ln\left(y^2-\sin^2x\right)dx.\notag
\end{equation}
при $y>1$. 

\noindent
Через J(1) выражается значение следующего интеграла:
$$
\int_{0}^{\pi/2}\ln\left(\cos x\right) dx
$$
Его, однако, можно взять более простым способом. Сделайте это и сравните получающиеся ответы.

\vspace{15pt}
\noindent \textbf{Упражнение 2}

\noindent Вычислите интеграл, сведя его к $B$-функции
\begin{equation}
I(n,m)=\int_{0}^{\pi/2}\sin^{n}x\cos^{m}xdx.\notag
\end{equation}

\vspace{15pt}
\noindent \textbf{Упражнение 3}

\noindent Вычислите интеграл, сведя его к $B$-функции

\begin{equation}
I(n,m,k)	=\int_{0}^{\infty}\frac{x^{n}}{(x^{m}+1)^{k}}dx.\notag
\end{equation}

\noindent Когда он сходится?

\vspace{15pt}
\noindent \textbf{Упражнение 4}

\noindent Вычислите точно интегралы
\begin{equation}
I_{1}(\lambda)	=\int_{0}^{\infty}\frac{x\sin(\lambda x)}{x^{2}+1}dx,\notag
\end{equation}
\vspace{15pt}
\begin{equation}
I_{2}(\lambda)	=\int_{0}^{\infty}\frac{x\sin(\lambda x)}{(x^{2}+1)^{2}}dx.\notag
\end{equation}

\vspace{15pt}
\noindent \textbf{Задача 1}

\noindent Вычислите сведением к $\Gamma$-функции и к $B$-функции

\begin{equation}
I(m)	=\int_{0}^{\infty}\frac{\cos x}{x^{m}}dx \notag
\end{equation}
при $0<m<1$.

\vspace{15pt}
\noindent \textbf{Задача 2}

\noindent Распределение Ферми определяется как

\begin{equation}
n_{F}(\epsilon)	=\frac{1}{e^{\epsilon/T}+1},\notag
\end{equation}
где $T>0$ - температура.

\noindent Вычислите интеграл, зависящий от двух параметров
\begin{equation}
I(V,T)	=\int_{-\infty}^{\infty}d\epsilon\left(n_{F}(\epsilon-eV)-n_{F}(\epsilon)\right)\notag
\end{equation}
\noindent Такой интеграл возникает при вычислении электрического тока через туннельный контакт.

\noindent Почему не работает следующее рассуждение?

\noindent Разобьем интеграл на два:

\begin{equation}
I(V,T)	=\int_{-\infty}^{\infty}d\epsilon n_{F}(\epsilon-eV)-\int_{-\infty}^{\infty}d\epsilon n_{F}(\epsilon)\notag
\end{equation}

\noindent и сделаем сдвиг в первом интеграле $\epsilon'=\epsilon-eV$. Тогда

\begin{equation} 
I(V,T)	=\int_{-\infty}^{\infty}d\epsilon'n_{F}(\epsilon')-\int_{-\infty}^{\infty}d\epsilon n_{F}(\epsilon)=0 \notag
\end{equation}

\noindent \textit{Замечание: если немного подумать, ответ на задачу можно получить совсем без вычислений.}
\end{document}
