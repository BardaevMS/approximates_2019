\documentclass[a4paper,12pt]{article}

%%	Fonts
\usepackage{cmap}
\usepackage{mathtext}
\usepackage{hyperref}
\usepackage[T2A]{fontenc}
\usepackage[utf8]{inputenc}
\usepackage[english,russian]{babel}
\usepackage[scaled]{helvet}
%\usepackage{fullpage}

% 	Math packages
\usepackage{amsmath,amsfonts,amssymb,amsthm,mathtools}
\usepackage{icomma}
\usepackage{euscript}
\usepackage{mathrsfs}
\usepackage{tensor}
\usepackage{physics}
\usepackage{stackrel}
% 	My math operartors
%\mathtoolsset{showonlyrefs=true}
\newcommand{\bigo}[1]   {\,\ensuremath{\mathop{}\mathopen{}\mathcal{O}\mathopen{}\displaystyle\left(#1\right)}}
\newcommand{\smallo}[1]	{\,\scriptstyle\mathop{}\mathopen{}\mathcal{O}\mathopen{}\displaystyle\left(#1\right) }
\DeclareMathOperator{\arcsinh}{arcsinh}
\DeclareMathOperator{\arctanh}{arctanh}
\newcommand*{\hm}[1]{#1\nobreak\discretionary{} {\hbox{$\mathsurround=0pt #1$}}{}}

%%	Graphics
\usepackage{graphicx}
\graphicspath{{images/}}
\setlength\fboxsep{3pt}
\setlength\fboxrule{1pt}
\usepackage{wrapfig}
% 	Tables
\usepackage{array,tabularx,tabulary,booktabs}
\usepackage{longtable}
\usepackage{multirow}
\usepackage{caption}
%\usepackage{subcaption}
\usepackage{subfig}
%	Itemize
\usepackage{enumitem}
\setlist[itemize]{topsep=0pt, leftmargin=0.2in}

%% 	Theorems
\theoremstyle{plain} % Это стиль по умолчанию, его можно не переопределять.
 \newtheorem{theorem}{Теорема}[section]
 \newtheorem{proposition}{Утверждение}
%\newtheorem{proposition}[theorem]{Утверждение}
 \newtheorem{lemma}{Лемма}
\theoremstyle{definition} % "Определение"
 \newtheorem{definition}{Определение}[section]
 \newtheorem{corollary}{Следствие}[theorem]
 \newtheorem{problem}{Задача}[section]
\theoremstyle{remark} % "Примечание"
 \newtheorem*{solution}{Решение}
\renewcommand\qedsymbol{$\blacksquare$}
\newcommand{\proofbegin}{\ensuremath{\blacktriangle}\nopunct}
\newcommand{\contbegin}{\footnotesize{\textcircled{\scriptsize!}}\normalsize \ }
\newcommand{\contend}{\ensuremath{\otimes}}
\newcommand{\vect}[1]{\ensuremath{\mathbf{#1}}}

%%	Programming
\usepackage{etoolbox} % логические операторы

%%	Page
\usepackage[14pt]{extsizes} % Возможность сделать 14-й шрифт
\usepackage{geometry} % Простой способ задавать поля
	\geometry{top=20mm}
	\geometry{bottom=20mm}
	\geometry{left=12mm}
	\geometry{right=12mm}
	\geometry{bindingoffset=0mm}
%\linespread{0.01}
%\usepackage{mathpazo}
%usepackage{fancyhdr} % Колонтитулы
% 	\pagestyle{fancy}
% 	\renewcommand{\headrulewidth}{.1mm}  % Толщина линейки, отчеркивающей верхний колонтитул
% 	\lfoot{}
% 	\rfoot{}
% 	\rhead{}
% 	\chead{}
% 	\lhead{}
 	% \cfoot{Нижний в центре} % По умолчанию здесь номер страницы
\usepackage{setspace} % Интерлиньяж
\onehalfspacing % Интерлиньяж 1.5
%\doublespacing % Интерлиньяж 2
%\singlespacing % Интерлиньяж 1
%\linespread{0.01}
\usepackage{lastpage} % Узнать, сколько всего страниц в документе.
\usepackage{soul} % Модификаторы начертания
%\usepackage{hyperref}
\usepackage[usenames,dvipsnames,svgnames,table,rgb]{xcolor}
\hypersetup{					% Гиперссылки
    unicode=true,           	% русские буквы в раздела PDF
    pdftitle={},   				% Заголовок
    pdfauthor={Nestyuk Arseniy},	% Автор
    pdfsubject={},      		% Тема
    pdfcreator={Nestyuk Arseniy}, 	% Создатель
    pdfproducer={}, 			% Производитель
    pdfkeywords={Asymptotic Analysis} {Integral} {Gamma-function}, % Ключевые слова
    colorlinks=true,       		% false: ссылки в рамках; true: цветные ссылки
    linkcolor=red,          	% внутренние ссылки
    citecolor=green,        	% на библиографию
    filecolor=magenta,      	% на файлы
    urlcolor=cyan           	% на URL
}
\usepackage{multicol} % Несколько колонок

% 	Titel
\author{}
\title{}
\date{\today}

\graphicspath{ {images/} }


\begin{document}
\section*{Задание к семинару по Wolfram Mathematica}

\subsection*{Упражнение 1: разминка}
Вспомните упражнение 4 семинара 2, где мы заменяли сумму
\begin{equation}
\sum_{n=0}^{\infty} n^a e^{-bn}
\end{equation}
на интеграл. Вычислите сумму и интеграл точно для значений $a=1.1,\ b=0.1$, и проверьте, насколько оправдана такая замена. Сумма и интеграл должны быть написаны значками суммы и интеграла, а не функциями Sum and Integrate.

\subsection*{Упражнение 2: поиск нужных корней}
В задаче 2 семинара 1 мы искали длины промежутков, на которых выполняется
\begin{equation}
	\left| \cos x + \alpha \frac{\sin x}{x} \right| > 1.
\end{equation}
Найдите их численно для некоторого малого $\alpha$.

\subsection*{Упражнение 3: метод перевала}
Напишите функцию, которая принимает один аргумент, функцию $f(x)$, и возвращает вычисленный методом перевала интеграл
\begin{equation}
	\int_{-\infty}^{\infty} e^{f(x)} \dd x.
\end{equation}
Проверять условия применимости метода перевала не надо; можно считать, что все вторые производные в перевальных точках отличны от нуля; можно считать, что перевальных точек конечное количество.

\subsection*{Упражнение 4: асимптотические разложения}
В упражнениях 3 и 4 семинара 5 были найдены асимптотические ряды для интегрального косинуса и неполной гамма-функции соответственно. Выберите ту из этих функций, для которой вы нашли асимптотическое разложение (если не нашли, идите и сдайте пятый семинар, а потом возвращайтесь). Просуммируйте асимптотический ряд до минимального члена и постройте результат на одном графике с функцией. 

Найдите разницу между рядом исходной функцией и рядом, просуммированным до минимального члена, для некоторого большого $x$. Найдите эту разницу для только первого (нескольких первых) членов разложения; для ряда, просумированного дальше, чем до минимального члена. Сделайте выводы и сообщите их преподавателю.

\subsection*{Задача 1: Задача трёх тел}
Классическая задача небесной механики состоит в определении траекторий движения трёх тел, притягивающихся по закону Ньютона. Таки это проблема: в общем случае задача не имеет аналитических решений. Однако картинки обычно получаются интересные, поэтому сегодня мы будем решать её численно. Посмотрите на \href{https://ru.wikipedia.org/wiki/%D0%97%D0%B0%D0%B4%D0%B0%D1%87%D0%B0_%D1%82%D1%80%D1%91%D1%85_%D1%82%D0%B5%D0%BB}{статью в Википедии}. Справа показывают мультик, от вас я жду такого же (не считая указания центра масс).

Для упрощения решения все массы будем считать равными 1. Гравитационная постоянная, конечно, тоже равна 1. Найдите численное решение системы дифференциальных уравнений для трёх тел, которые изначально покоились, находясь в точках (0,0), (0,1) и (2,0).

Hints: Три точки, очевидно, всегда находятся в одной плоскости. Гравитационное притяжение их из этой плоскости не выводит. Поэтому все радиус-вектора можно считать двумерными. 

Основная функция определяется словами ``численное решение диффуров''. Она прекрасно работает с системами уравнений. Обратите внимание на то, что в правой части уравнений шесть раз повторяется относительно громоздкая конструкция из разности векторов, делённой на модуль разности в степени; если вы создадите функцию двух векторов, возвращающую эту конструкцию, количество кода значительно уменьшится. Построить график параметрически заданных функций можно с помощью ParametricPlot. Показ мультиков в математике мы разбирали на семинаре.

\subsection*{Задача 2: Параметрический резонанс}

В семинаре про преобразование Фурье была разобрана задача (2) про маятник (или грузик на пружинке, или осциллятор), который возмущается периодической внешней силой 
\begin{equation}
	f(t) = \begin{cases}
		-f_0,& -T/2<t<0, \\
		f_0,&  0<t<T/2.
	\end{cases}
\end{equation}

Мы обнаружили, что если частота маятника совпадает с одним из значений $2\pi n/T$, то наблюдается параметрический резонанс и неограниченный рост амплитуды колебаний, линейный по времени. Конечно, при достаточно большой амплитуде предположение, положенное в основу решения, $\sin \phi \approx \phi$, неверно.

Решите точное уравнение численно и посмотрите, когда заканчивается рост амплитуды и что происходит дальше.

\end{document}