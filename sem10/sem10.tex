\documentclass[a4paper,12pt]{article}
\usepackage{graphicx}
\usepackage{bm,amssymb}
\usepackage{mathrsfs}
\usepackage[unicode,colorlinks=true,filecolor=blue, menucolor=black, linkcolor=black, citecolor=black,pagebackref=white]{hyperref}
\usepackage[utf8]{inputenc}
\usepackage[russian]{babel}
\usepackage{amsmath}
\usepackage{feynmp}
\usepackage{caption}
\usepackage[left=2cm,right=2cm, top=2cm,bottom=2cm,bindingoffset=0cm]{geometry}
\begin{document}
\title{Семинар по теме: <<Теория возмущений>>}
\maketitle

\subsection*{Обозначения}

Линейные операторы (то есть матрицы) мы будем обозначать как $\hat{H}$:

\[
\hat{H}=\begin{pmatrix}1 & 5 & 4\\
5 & 1 & 3\\
4 & 3 & 2
\end{pmatrix}
\]

\noindent
Вектора линейного пространства, на которые эти операторы действуют,
мы будем обозначать как $\left|a\right>$:
\[
\left|a\right\rangle =\begin{pmatrix}1\\
2\\
0
\end{pmatrix}
\]


\noindent
Эрмитово сопряжённые вектора (операция эрмитового сопряжения - это
транспонирование и комплексное сопряжение) мы будем обозначать как
$\left<b\right|=(\left|b\right>)^{\dagger}$:
\[
\left|b\right\rangle =\begin{pmatrix}5\\
1\\
3
\end{pmatrix}\Rightarrow\left<b\right|=\begin{pmatrix}5 & 1 & 3\end{pmatrix}
\]


\noindent
Таким образом, например, действие оператора на вектор обозначается
как:
\[
\hat{H}\left|a\right>=\begin{pmatrix}1 & 5 & 4\\
5 & 1 & 3\\
4 & 3 & 2
\end{pmatrix}\begin{pmatrix}1\\
2\\
0
\end{pmatrix}=\begin{pmatrix}11\\
7\\
10
\end{pmatrix}
\]


\noindent
и скалярное произведение обозначается как:
\[
\left\langle b|a\right\rangle =\begin{pmatrix}5 & 1 & 3\end{pmatrix}\begin{pmatrix}1\\
2\\
0
\end{pmatrix}=7
\]


\noindent
Эти обозначения пришли из квантовой механики, в которой чаще всего
и применяется алгебраическая теория возмущений. 


\subsection*{Общие сведения}

Говорят, что вектор $\left|a\right\rangle $ - собственный вектор
для оператора $\hat{H}$, сооветствующий собственному числу $\lambda$,
если: 
\[
\hat{H}\left|a\right\rangle =\lambda\left|a\right\rangle 
\]


\noindent
Матрица $\hat{H}$ называется эрмитовой, если $\hat{H}^{\dagger}=\hat{H}$.
В частности, если матрица вещественна, то эрмитовость для неё означает
симметричность. Известно, что для любого эрмитового оператора можно
выбрать базис пространства, состоящий из его собственных векторов.
Это эквивалентно утверждению о том, что существует базис $\{\left|n\right\rangle \}_{n=1}^{N}$,
такой, что матрица $\hat{H}$, записанная в этом базисе диагональна:

\[
H_{nm}\equiv\left\langle n\left|\hat{H}\right|m\right\rangle ={\rm diag}(\lambda_{1},\dots,\lambda_{N})
\]


\noindent
Этот базис можно выбрать ортонормированным, так что:
\[
\left\langle n|m\right\rangle =\delta_{nm}=\begin{cases}
1, & n=m\\
0, & n\neq m
\end{cases}
\]

\subsection*{Теория возмущений}


\subsubsection*{Постановка задачи}

Пусть имеется линейный оператор $\hat{H}_{0}$, для которого известен
ортонормированный базис $\left\{ \left|n^{(0)}\right\rangle \right\} _{n=1}^{N}$
из его собственных векторов:
\[
\hat{H}_{0}\left|n^{(0)}\right\rangle =E_{n}^{(0)}\left|n^{(0)}\right\rangle 
\]


\noindent
Теория возмущений решает задачу о (приближенном) нахождении собственных
векторов и собственных значений матрицы $\hat{H}=\hat{H}_{0}+\epsilon\hat{V}$,
где параметр $\epsilon\ll1$, в виде разложения по малости параметра
$\epsilon$:
\[
\hat{H}\left|n\right\rangle =E_{n}\left|n\right\rangle 
\]


\noindent
при этом 
\[
E_{n}=E_{n}^{(0)}+\epsilon E_{n}^{(1)}+\epsilon^{2}E_{n}^{(2)}+\dots
\]


\[
\left|n\right>=\left|n^{(0)}\right>+\epsilon\left|n^{(1)}\right>+\epsilon^{2}\left|n^{(2)}\right>+\dots
\]



\subsubsection*{Невырожденный случай}

Если собственное число $E_{n}^{(0)}$ оказывается невырожденным (это
означает, что ему соответствует ровно один собственный вектор $\left|n^{(0)}\right>$),
то в первом порядке теории возмущений поправка к нему дается выражением:


\paragraph{
\[
E_{n}^{(1)}=V_{nn}\equiv\left\langle n^{(0)}\left|\hat{V}\right|n^{(0)}\right\rangle 
\]
}

\noindent
а во втором порядке теории возмущений - выражением:
\[
E_{n}^{(2)}=\sum_{k\neq n}\frac{V_{kn}V_{nk}}{E_{n}^{(0)}-E_{k}^{(0)}}=\sum_{k\neq n}\frac{\left|V_{nk}\right|^{2}}{E_{n}^{(0)}-E_{k}^{(0)}}
\]


\noindent
Поправки к собственному вектору в первом порядке теории возмущений
даются выражением:
\[
\left|n^{(1)}\right\rangle =\sum_{k\neq n}\frac{V_{kn}}{E_{n}^{(0)}-E_{k}^{(0)}}\left|k^{(0)}\right\rangle 
\]


\noindent
Видно, что если имеется случай, когда $V_{kn}\neq0$, но $E_{k}^{(0)}=E_{n}^{(0)}$,
то имеется проблема, связанная с делением на ноль. Это соответствует
вырождению собственного числа (то есть одному собственному числу соответствуют
два собственных вектора), и такие случаи нужно рассматривать отдельно.


\subsubsection*{Вырожденный случай}

Пусть собственное число $E_{n}^{(0)}$ является $s$-кратно вырожденным
(что означает, что среди набора $\left\{ \left|n^{(0)}\right\rangle \right\} _{n=1}^{N}$
имеются $s$ различных векторов $\left\{ \left|n_{k}^{(0)}\right\rangle \right\} _{k=1}^{s}$,
соответствующих этому собственному числу), то схема действия следующая.
Сперва запишем проекцию матрицы $\hat{V}$ на вырожденное собственное
подпространство. Это значит, что нужно рассмотреть матрицу $\hat{\widetilde{V}}$
размера $s\times s$, которая записывается как:
\[
\widetilde{V}_{ab}=\left\langle n_{a}^{(0)}\left|\hat{V}\right|n_{b}^{(0)}\right\rangle =\begin{pmatrix}V_{n_{1}n_{1}} & V_{n_{1}n_{2}} & \dots & V_{n_{1}n_{s}}\\
V_{n_{2}n_{1}} & V_{n_{2}n_{2}} & \dots & V_{n_{2}n_{s}}\\
\vdots & \vdots & \ddots & \vdots\\
V_{n_{s}n_{1}} & V_{n_{s}n_{2}} & \dots & V_{n_{s}n_{s}}
\end{pmatrix},
\quad V_{n_i n_j} \equiv \left\langle n_i^{(0)} \left|\hat{V}\right|n_j^{(0)}\right\rangle
\]


\noindent
Затем эту матрицу нужно диагонализовать стандартным образом в базисе
из $\left\{ \left|n_{k}^{(0)}\right\rangle \right\} _{k=1}^{s}$ (что
гораздо проще - исходная матрица была размера $N\times N$, а эта
матрица - размера $s\times s$; как правило, кратность вырождения
$s$ - не очень большое число).Для этого записывается секулярное уравнение
\[
\det(\hat{\widetilde{V}}-v\hat{\mathbb{I}})\equiv\det\begin{pmatrix}V_{n_{1}n_{1}}-v & V_{n_{1}n_{2}} & \dots & V_{n_{1}n_{s}}\\
V_{n_{2}n_{1}} & V_{n_{2}n_{2}}-v & \dots & V_{n_{2}n_{s}}\\
\vdots & \vdots & \ddots & \vdots\\
V_{n_{s}n_{1}} & V_{n_{s}n_{2}} & \dots & V_{n_{s}n_{s}}-v
\end{pmatrix}=0
\]


\noindent
затем находятся $s$ его собственных чисел $\left\{ v_{a}\right\} _{a=1}^{s}$
и $s$ его собственных векторов $\left\{ \left|\widetilde{n}_{k}\right\rangle \right\} _{a=1}^{s}$.
Эти вектора называются <<правильными векторами главного (ведущего)
приближения>>. Они были выбраны таким образом, что они тоже являются
собственными векторами исходного оператора $\hat{H}_{0}$, с собственным
числом $E_{n}^{(0)}$, и, кроме того, образуют базис вырожденного
собственного подпространства. Далее необходимо перейти от базиса исходных
векторов $\left\{ \left|n_{a}^{(0)}\right\rangle \right\} _{a=1}^{s}$
к базису из $\left\{ \left|\widetilde{n}_{a}\right\rangle \right\} _{a=1}^{s}$;
и в новом базисе уже можно применять стандартные формулы для невырожденной
теории возмущений. В частности, случая, когда происходит деление на
ноль ($V_{kn}\neq0$ но $E_{k}^{(0)}=E_{n}^{(0)}$) уже не будет.\\\\
Заметим, что при этом числа $\left\{ v_{a}\right\} _{a=1}^{s}$ (которые
являлись собственными числами матрицы $\widetilde{V}$) будут играть
роль первой поправки $E_{n}^{(1)}$ к собственному числу $E_{n}^{(0)}$;
кроме того, поскольку этих чисел $s$, и они в общем случае различны,
то говорят о снятии вырождения возмущением - число $E_{n}^{(0)}$
перестаёт быть вырожденным, происходит расщепление.


\paragraph{Замечание}

Параметр $\epsilon$ был введён лишь для того, чтобы аккуратно следить
за тем, какой порядок теории возмущений рассматривается. Оказывается,
что в $k$-м порядке теории возмущений, матрица $\hat{V}$ входит
ровно $k$ раз (и этот порядок домножается на $\epsilon^{k}$); это
позволяет нам формально положить параметр $\epsilon=1$ во всех выражениях,
и просто считать саму матрицу $\hat{V}$ малой.

\paragraph{Литература}
\cite{LL3}, §38 (``возмущения, не зависящие от времени'') и §39 (``секулярное уравнение'').

\subsection*{Задача 1}

Считая параметр $a>b$ и $\epsilon\ll1$, исследуем с помощью теории
возмущений матрицу 
\[
\hat{H}=\begin{pmatrix}a & \epsilon\\
\epsilon & b
\end{pmatrix}
\]



\subsubsection*{Невырожденная теория возмущений}

Невозмущенные собственные вектора и собственные значения матрицы $\hat{H}_{0}$
записываются тривиально как:
\[
\begin{cases}
\left|1\right\rangle  & =\begin{pmatrix}1\\
0
\end{pmatrix}\Rightarrow\lambda_{1}^{(0)}=a\\
\left|2\right\rangle  & =\begin{pmatrix}0\\
1
\end{pmatrix}\Rightarrow\lambda_{2}^{(0)}=b
\end{cases}
\]


\noindent
Следуя теории возмущений, первая поправка к собственным числам $\lambda_{1}^{(0)}$
записываются как:
\[
\begin{cases}
\lambda_{1}^{(1)} & =V_{11}=\left\langle 1\left|\hat{V}\right|1\right\rangle =0\\
\lambda_{2}^{(1)} & =V_{22}=\left\langle 2\left|\hat{V}\right|2\right\rangle =0
\end{cases}
\]


\noindent
Эти поправки оказались нулевыми, поэтому необходимо исследовать следующий
порядок теории возмущений. Он даёт нам:
\[
\begin{cases}
\lambda_{1}^{(2)} & =\sum_{k\neq1}\frac{\left|V_{k1}\right|^{2}}{\lambda_{1}^{(0)}-\lambda_{k}^{(0)}}=\frac{\left|V_{21}\right|^{2}}{a-b}=\frac{\epsilon^{2}}{a-b}\\
\lambda_{2}^{(2)} & =\sum_{k\neq2}\frac{\left|V_{k2}\right|^{2}}{\lambda_{2}^{(0)}-\lambda_{k}^{(0)}}=\frac{\left|V_{12}\right|^{2}}{b-a}=-\frac{\epsilon^{2}}{a-b}
\end{cases}
\]


\noindent
Таким образом, приближенно спектр записывается как:
\[
\begin{cases}
\lambda_{1} & \approx a+\frac{\epsilon^{2}}{a-b}\\
\lambda_{2} & \approx b-\frac{\epsilon^{2}}{a-b}
\end{cases}
\]

\subsubsection*{Вырожденная теория возмущений}
Попробуем решить эту задачу, пользуясь вырожденной теорией возмущений (положив $a=b$). Имеем:
$$
\det\begin{pmatrix}-\lambda & \epsilon\\
\epsilon & -\lambda
\end{pmatrix}=\lambda^2-\varepsilon^2=0
$$
Отсюда,
\[
\begin{cases}
\lambda_{1} & \approx a+\varepsilon\\
\lambda_{2} & \approx a-\varepsilon
\end{cases}
\]
\paragraph{Точное решение}

Эту задачу можно решить точно. Уравнение на собственные значения для
матрицы $\hat{H}$ записывается как:
\[
\det\left(\hat{H}-\lambda\right)=\det\begin{pmatrix}a-\lambda & \epsilon\\
\epsilon & b-\lambda
\end{pmatrix}=(a-\lambda)(b-\lambda)-\epsilon^{2}=0
\]
\[
\lambda_{1,2}=\frac{a+b\pm\sqrt{(a-b)^{2}+4\epsilon^{2}}}{2}
\]


\noindent
Представим ответ в виде разложения по $\epsilon$, считая, что $a-b\gg\varepsilon$:
\[
\lambda_{1,2}=\frac{1}{2}\left(a+b\pm(a-b)\sqrt{1+\frac{4\epsilon^{2}}{(a-b)^{2}}}\right)
\approx\frac{1}{2}\left(a+b\pm(a-b)\left(1+\frac{2\epsilon^{2}}{\left(a-b\right)^{2}}\right)\right)=\begin{cases}
a+\frac{\epsilon^{2}}{a-b}\\
b-\frac{\epsilon^{2}}{a-b}
\end{cases}
\]
Видно, что воспроизвёлся ответ невырожденной теории возмущений.
\noindent
А теперь пусть $|a-b|\ll\varepsilon\ll1$:
\[
\lambda_{1,2}\frac{1}{2}(a+b)\pm\varepsilon
\]
Это воспроизводит в главном приближении ответ вырожденной теории возмущений (с точностью до членов порядка $a-b\ll\varepsilon$). Таким образом, переход от невырожденной теории возмущений к вырожденной происходит при $\varepsilon\sim a-b$. Действительно, именно тогда поправка невырожденной теории возмущений становится порядка $1$, т.е. перестаёт быть малой.


\subsection*{Задача 2 (эффект Штарка)}

В атоме водорода энергетические уровни нумеруются тремя квантовыми
числами $n$, $l$ и $m$. При этом число $n=1,2,\dots$,; при фиксированном
$n$, число $l=0,1,\dots,n-1$, а при фиксированных $n$ и $l$, число
$m=-l,\dots,l$. Энергия же состояния атома водорода зависит только
от числа $n$ (и выражается как $E=-\frac{{\rm Rd}}{n^{2}}$, где
${\rm Rd}$ называется постоянной Ридберга); тем самым, состояние
с $n=2$ оказывается четырёхкратно вырожденным по энергии (энергии
$E_{2}=-\frac{{\rm Rd}}{4}$ соответствуют состояния $\left|n,l,m\right\rangle \in\left\{ \left|2,0,0\right\rangle ,\left|2,1,-1\right\rangle ,\left|2,1,0\right\rangle ,\left|2,1,1\right\rangle \right\} $.
Наложение электрического поля воспринимается в этой задаче как возмущение;
при этом возмущенный оператор энергии (гамильтониан), записывается
как:
\[
\hat{H}=\begin{pmatrix}E_{2} & V & 0 & 0\\
V^{*} & E_{2} & 0 & 0\\
0 & 0 & E_{2} & 0\\
0 & 0 & 0 & E_{2}
\end{pmatrix}
\]


\noindent
Собственные значения этого оператора в квантовой механике играют роль
допустимых значений энергии системы. Требуется найти поправки при
наложении такого возмущения.


\subsubsection*{Решение}

В данном случае у невозмущенного оператора $\hat{H}_{0}$ имеется
четырехкратно вырожденный уровень энергии $E_{2}$, и имеются четыре
собственных вектора:
\[
\left|1^{(0)}\right\rangle =\begin{pmatrix}1\\
0\\
0\\
0
\end{pmatrix},\left|2^{(0)}\right\rangle =\begin{pmatrix}0\\
1\\
0\\
0
\end{pmatrix},\left|3^{(0)}\right\rangle =\begin{pmatrix}0\\
0\\
1\\
0
\end{pmatrix},\left|4^{(0)}\right\rangle =\begin{pmatrix}0\\
0\\
0\\
1
\end{pmatrix}
\]


\noindent
Поскольку имеется вырождение, то необходимо применять вырожденный
случай теории возмущений. Следуя ему, необходимо записать секулярное
уравнение:
\[
\det\left(\hat{V}-v\right)=\det\begin{pmatrix}-v & V & 0 & 0\\
V^{*} & -v & 0 & 0\\
0 & 0 & -v & 0\\
0 & 0 & 0 & -v
\end{pmatrix}
= v^{2}\left(v^{2}-\left|V\right|^{2}\right)=0\Rightarrow\begin{cases}
v_{1,2} & =\pm\left|V\right|\\
v_{3,4} & =0
\end{cases}
\]


\noindent
Далее, необходимо найти собственные вектора, соответствующие этим
собственным значениям. Пусть $V=\left|V\right|e^{i\varphi}$ ($V$
- комплексное число). Тогда, уравнения на собственные вектора записываются
как:
\[
\begin{pmatrix}-\left|V\right| & \left|V\right|e^{i\varphi} & 0 & 0\\
\left|V\right|e^{-i\varphi} & -\left|V\right| & 0 & 0\\
0 & 0 & -\left|V\right| & 0\\
0 & 0 & 0 & -\left|V\right|
\end{pmatrix}\begin{pmatrix}m_{11}\\
m_{12}\\
m_{13}\\
m_{14}
\end{pmatrix}=0
\Rightarrow\left|\widetilde{1}^{(0)}\right\rangle =\frac{1}{\sqrt{2}}\begin{pmatrix}e^{i\varphi}\\
1\\
0\\
0
\end{pmatrix}
\]


\[
\begin{pmatrix}\left|V\right| & \left|V\right|e^{i\varphi} & 0 & 0\\
\left|V\right|e^{-i\varphi} & \left|V\right| & 0 & 0\\
0 & 0 & \left|V\right| & 0\\
0 & 0 & 0 & \left|V\right|
\end{pmatrix}\begin{pmatrix}m_{21}\\
m_{22}\\
m_{23}\\
m_{24}
\end{pmatrix}=0
\Rightarrow\left|\widetilde{2}^{(0)}\right\rangle =\frac{1}{\sqrt{2}}\begin{pmatrix}e^{i\varphi}\\
-1\\
0\\
0
\end{pmatrix}
\]


\noindent
и тривиально $\left|\widetilde{3}^{(0)}\right\rangle =\begin{pmatrix}0\\
0\\
1\\
0
\end{pmatrix}$ и $\left|\widetilde{4}^{(0)}\right\rangle =\begin{pmatrix}0\\
0\\
0\\
1
\end{pmatrix}$. Напомним, вектора $\left|\widetilde{k}^{(0)}\right\rangle $ называются
<<правильными векторами ведущего приближения>>, и далее необходимо
перейти к базису из этих векторов.В частности, первый порядок теории
возмущений даст поправки, совпадающиес собственными числами $v_{k}$:
\[
\begin{cases}
E_{2;1} & =E_{2}-\left|V\right|\\
E_{2;2} & =E_{2}+\left|V\right|\\
E_{2;3,4} & =E_{2}
\end{cases}
\]


\noindent
Таким образом, уже в первом порядке теории возмущений, вырождение
частично снялось (вместо четырехкратно вырожденного уровня энергии
$E_{2}$ мы получаем однократно вырожденный уровень энергии $E_{2}+|V|$,
однократно вырожденный уровень $E_{2}-|V|$ и двукратно вырожденный
уровень $E_{2}$). Кроме того, расщепление линейно по возмущению $|V|$;
это - прямое следствие вырождения в этой задаче. В квантовой механике
это явление называется линейным эффектом Штарка. Если бы вырождения
не было, то ведущая поправка была бы лишь во втором порядке теории
возмущений, и поправки к уровням энергии были бы квадратичны по $|V|$
(как в первой задаче).


\subsection*{Задачи для домашнего решения}

\noindent \textbf{Упражнение 1}

\noindent Рассматриваются эрмитовы матрицы (операторы) $\hat{H}$ и $\hat{V}$ размера $n\times n$, при этом все собстенные значения $E_{1}^{0},...,E_{n}^{0}$ и нормированные собственные векторы $|\psi_{1}^{0}\rangle,...,|\psi_{n}^{0}\rangle$ матрицы $\hat{H}$ считаются известными, более того $E_{i}^{0}\neq E_{j}^{0}$ при $i\neq j$ (невырожденный случай). Далее составляется матрица $\hat{H}+\epsilon\hat{V}$. При $\epsilon\rightarrow 0$, ищем собственные векторы и значения этой матрицы в виде
\begin{equation}\notag
E_{i}	=E_{i}^{0}+\epsilon E_{i}^{1}+\epsilon^{2}E_{i}^{2}+...
\end{equation}
\begin{equation}
\notag
|\psi_{i}\rangle	=|\psi_{i}^{0}\rangle+\epsilon|\psi_{i}^{1}\rangle+\epsilon^{2}|\psi_{i}^{2}\rangle+...
\end{equation}
\noindent На семинаре было доказано, что $E_{i}^{1}=\langle\psi_{i}^{0}|V|\psi_{i}^{0}\rangle$. Получите формулы для $|\psi_{i}^{1}\rangle$ и для $E_{i}^{2}$. Не забудьте, что мы рассматриваем нормированные векторы, т.е. при выводе есть дополнительное условие $\langle\psi_{i}|\psi_{j}\rangle=\langle\psi_{i}^{0}|\psi_{j}^{0}\rangle=\delta_{ij}$.

\vspace{15pt}
\noindent \textbf{Упражнение 2}

\noindent Матрицы Паули определены как
\begin{equation}\notag
\sigma_{x}	=\left(\begin{array}{cc}
0 & 1\\
1 & 0
\end{array}\right),\quad\sigma_{y}=\left(\begin{array}{cc}
0 & -i\\
i & 0
\end{array}\right),\quad\sigma_{z}=\left(\begin{array}{cc}
1 & 0\\
0 & -1
\end{array}\right)
\end{equation}
\noindent Для матрицы $\hat{H}+\epsilon\hat{V}$ найдите поправки к собстевнным значениям вплоть до второго порядка по $\epsilon$ и к собственным векторам вплоть до первого порядка для следующих случаев: 
\begin{equation}\notag
	\hat{H}=\sigma_{x},\quad\hat{V}=\sigma_{x};\quad\hat{H}=\sigma_{x},\quad\hat{V}=\sigma_{y};\quad\hat{H}=\sigma_{x},\quad\hat{V}=\sigma_{z}
\end{equation}

\vspace{15pt}
\noindent \textbf{Упражнение 3}

\noindent Матрицы $\hat{H}$ и $\hat{V}$ имеют вид:

\begin{equation}\notag \hat{H}	=\left(\begin{array}{ccc}
2 & 0 & 1\\
0 & 2 & 0\\
1 & 0 & 2
\end{array}\right),\quad\hat{V}=\left(\begin{array}{ccc}
0 & 3 & 0\\
3 & 2 & 1\\
0 & 1 & 0
\end{array}\right).
\end{equation}
\noindent Используя теорию возмущений, найдтите с точностью до $\epsilon^{2}$ собственные значения матрицы $H+\epsilon V$, $\epsilon\rightarrow 0$ (собственные векторы искать не нужно).

\vspace{15pt}
\noindent \textbf{Задача 1 (Большая задача про квантовую механику)}

\noindent \textbf{\textit{Часть один}}

\noindent Любой дифференциальный оператор $\hat{H}$ можно рассматривать, как матрицу в функциональном пространстве. Т.е. для обычных матриц $n\times n$ при действии матрицы на вектор получался какой-то другой вектор: $\hat{H}|\psi\rangle=|\psi'\rangle$. Точно также, дифференциальный оператор $\hat{H}$, действуя на какую-нибудь функцию одной переменной $\psi(x)$, возвращает какую-то другую функцию $\psi'(x)$: 
\begin{equation}\notag
\hat{H}\psi(x)	=\left(a_{0}(x)+a_{1}(x)\frac{d}{dx}+a_{2}(x)\frac{d^{2}}{dx^{2}}+...+a_{N}(x)\frac{d^{N}}{dx^{N}}\right)\psi(x)=\psi'(x).
\end{equation}
\noindent Дальше мы будем рассматривать оператор
\begin{equation}\notag
\hat{H}	=-\frac{d^{2}}{dx^{2}}-\kappa\delta(x),
\end{equation}
\noindent где $\kappa>0$.

\noindent Собственные функции (векторы) такого оператора определены обычным образом
\begin{equation}\notag
\hat{H}\psi(x)	=E\psi(x).
\end{equation}
\noindent Это уравнение на совбственные значения и функции в квантовой механики называется стационарным уравнением Шредингера. Скалярное произведение определено как
\begin{equation}\notag
(\psi(x)\cdot\phi(x))	=\int_{-\infty}^{\infty}dx\psi^{*}(x)\phi(x).
\end{equation}
\noindent Эта формула очень сильно напоминает формулу для $n$-компонентных векторов: $(\vec{\psi}\cdot\vec{\phi})=\langle\psi|\phi\rangle=\sum_{i=1}^{n}\psi_{i}^{*}\phi_{i}$. Найдите собственную функцию $\hat{H}$, которая спадает в ноль на $\pm\infty$ и соответствующее ей собстенное значение $E$. Тут нужно действовать так:
\begin{itemize}
\item Найдите общее решение уравнения Шредингера при $x>0$ и $x<0$ по отдельности для произвольного $E$. В итоге у вас получится 4 неизвестных коэффициента, которые нужно определить, а также определить нужно $E$. Теперь потребуем убывания функций на бесконечностях. Какие условия это требование накладывает на неизвестные коэффициенты и на величину $E$? 

\item Теперь нужно “обработать” точку $x=0$. Первое условие, которое накладывается на нашу функцию в этой точке - непрерывность. Чтобы получить еще одно условие, которое позволит явно найти $E$ необходимо учесть наличие $\delta$-функции в операторе. Чтобы получить это условие, проинтегрируйте уравнение Шредингера в пределах от $-\lambda$ до $\lambda$, а затем возьмите предел $\lambda\rightarrow+0$

\item После всех этих процедур собственная функция найдена с точностью до общего множителя, а собстенное число $E$ - определено. Отнормируйте собственную функцию на единицу, т.е. потребуйте $(\psi(x)\cdot\psi(x))=1$.
\end{itemize}
\noindent \textit{\textbf{Часть два}}

\noindent \textit{Найдите поправку к найденному собственному значению, получающуюся при рассмотрении слабо отличающегося дифференциального оператора} $\hat{H}+\epsilon\hat{V}$, где $\hat{V}=\sum_{i=-\infty}^{\infty}\delta(x-i)$, $\epsilon\rightarrow 0$.

\noindent Оказывается, что все остальные собственные функции для исходного оператора $\hat{H}$ (а их бесконечно много) имеют другой знак $E$, т.е. ситуация невырожденная. Тогда можно воспользоваться теорией возмущений для матриц. Как всегда, ищем собстевнное значение полного оператора в виде $E=E^{0}+\epsilon E^{1}+...$, где $E^{0}$ - собстенное значение $\hat{H}$ определенное в первой части задачи. Тогда $E^{1}$ можно найти по формуле
\begin{equation}\notag
(\psi\cdot\hat{V}\psi)	=\int_{-\infty}^{\infty}dx\psi(x)\hat{V}\psi(x)
\end{equation}
\noindent где $\psi(x)$ также был найден в первой части задачи.

\noindent 
\textbf{Задача 2 (операторные функции Грина и теория возмущений)}

\noindent
Для случая невырожденных собственных значений существует несколько другая техника получения поправок к энергии. Центральный объект в этой технике - операторная функция Грина, определяемая следующим образом:
$$
\hat{G}(E)=\left(E-\hat{H}\right)^{-1}
$$
Рассмотрим матричный элемент $\langle n |\hat{G}^{-1}| n \rangle=E-E_n$ (собственные вектора нормированы на 1). Как легко видеть, он равен $0$ при $E\to E_n$, т.е. в точке, отвечающей спектру системы. Это свойство мы и будем использовать в дальнейшем.

\noindent
\textbf{1.} Как обычно, пусть $\hat{H}=\hat{H_0}+\hat{V}$, где $V$ мало. Докажите следующее разложение для $\hat{G}(E)$: 
$$
\hat{G}(E)=\hat{G}_0(E)+\hat{G}_0(E)\hat{V}\hat{G}_0(E)+\hat{G}_0(E)\hat{V}\hat{G}_0(E)\hat{V}\hat{G}_0(E)+...
$$
\textbf{2.}
\end{document}
